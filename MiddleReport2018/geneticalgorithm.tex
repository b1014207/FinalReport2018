\section{遺伝的アルゴリズム}
\subsection{生物進化メカニズムの概要}

遺伝的アルゴリズムは進化計算とも呼ばれる,生物の進化メカニズムを応用した
近似最適化アルゴリズムである。環境に適応している生物は生存しつつ増殖する。
増殖には,生物自身の情報を保有している遺伝子が介在している。

単細胞生物をはじめとする下等生物は主に体細胞分裂を行うことで増殖する。体細胞分裂(または無性生殖)では
生物が持つ自信の設計情報,すなわち遺伝子が保有する情報は分裂した後の個体と前の個体とで
完全に一致する。

一方で人間を含む高等生物では,主に有性生殖によって増殖する。体細胞分裂は単一の個体で行われる
のに対して,有性生殖では2つの個体によって行われる。有性生殖を行う2個体の両方の遺伝子から
新たに誕生する個体の遺伝子が再生成され,親個体と子個体とで違った遺伝情報を保持するようになる。

以上の無性生殖,有性生殖による遺伝情報の伝達以外にも個体の遺伝情報を変化させ得る要因が存在する。
生物個体の完全なコピーを作る場合でも,有性生殖によって遺伝子を組み合わせる場合でも,遺伝子の複写
作業は物理的な空間で行われるため,化学的な要因や放射線による破壊等で遺伝情報が元の個体生成プロセス
とは別に変化することがある。これを突然変異という。突然変異は生殖時のエラーとも言えるが,生物の進化においては
重要な要素の一つとなっている。

増殖の過程は大まかに以上のような過程で親個体から子個体が発生するが,もう一つの重要な要素が自然淘汰である。
どの個体も同じように生存し,子孫を残すわけではなく,環境に適応できない個体および遺伝形質は長い時間をかけて
淘汰される。結果として比較的環境に適応している個体群,遺伝形質が存続するようになる。これを自然淘汰という。

遺伝的アルゴリズムは自然界に存在する生殖のメカニズム,自然淘汰のメカニズムを計算に応用した
アルゴリズムである。

  \bunseki{※米村祥裕}
\subsection{遺伝子の構造}
生物の遺伝情報はDNAによって保持され,DNAは主に4種類の塩基によって構成される。
DNAは塩基の並び方によって遺伝情報を表現する。一つの遺伝子座(塩基配列上の位置)における
遺伝子の組み合わせを遺伝子型(genotype)という。ある遺伝子によってほぼ決定される形質のことを
遺伝子の表現型(phenotype)という。表現型は厳密に遺伝子によってのみ決定されるわけではなく,
他の遺伝子,環境などによっても変化しうる。

  \bunseki{※米村祥裕}
\subsection{単純遺伝的アルゴリズム}
前節までで説明した進化メカニズムを基にして,考案されたのが単純遺伝的アルゴリズム(Simple Genetic Algorithm)である。
単純遺伝的アルゴリズムは主に次の処理からなる。

\begin{itemize}
  \item 符号化(遺伝子型の決定,コーディングという)
  \item 環境(問題の要件)に対する適応度の算出
  \item 選択(Selection)
  \item 交叉(Cross Over)
  \item 突然変異(Mutation)
\end{itemize}

  \bunseki{※米村祥裕}
\subsubsection{符号化}
遺伝子の構造については先に説明したように,遺伝子座上の4種類の塩基の並びによって情報が記述される。
単純遺伝的アルゴリズムの場合も同様に,対象とする問題を2進ビット列に置き換える。まずそれぞれの遺伝子座を
表現型と対応させる必要がある。表現型は例えば,特定の文字列の生成にアルゴリズムを適用する場合は文字に相当する。
例として英字アルファベットからなる特定の文字列を単純遺伝的アルゴリズムで生成する場合は,文字をビット列に置き換える。
英字アルファベットは26文字であるから2進数表現にした場合に必要なビット数は1文字あたり5ビットとなる。さらに文字列長を$\b{n}$とすると
遺伝子座数は$n$個となるので,1個体を表現するのに必要な情報量は$5\b{n}$ビットとなる。

  \bunseki{※米村祥裕}
\subsubsection{適応度の算出}
生成された個体が問題に対してどの程度適応しているかを算出するために,まずは個体の遺伝情報,すなわち符号化によって
得られたビット列を表現型に変換する。符号化の項目で挙げた文字列生成の例では,文字列を構成する英字アルファベット1文字当たり
5ビットとなっているので,ビット列全体を5ビットずつに分け,分けた5ビットごとに英字アルファベットに変換する。
そして,一致している要素の個数等で問題の解により近い個体の適応度が高くなるように計算を行う。

適応度の算出は,アルゴリズムを適用する問題に応じて異なる。

  \bunseki{※米村祥裕}
\subsubsection{選択}
複数の個体からなる集団から相対的により良い個体を次世代に残すことでより最適な解に近づける。
基本的には適応度が高い個体が残りやすいように選択を行う。選択のアルゴリズムとしてはルーレット法(適応度比例選択)や
トーナメント法,エリート選択法,ランキング法等が用いられる。

  \bunseki{※米村祥裕}
\subsubsection{ルーレット法(適応度比例選択)}
集団$P$中のある個体$i$が選択される確率$p_i$を個体$i$の適応度$f_i$を基に式で算出される。

\begin{equation}
  p_i = \frac{f_i}{\sum_{j\in P} f_j}
\end{equation}

適応度に比例して選択される確率が増加するため,適応度比例選択と呼ばれる。

  \bunseki{※米村祥裕}
\subsection{交叉}
交叉は有性生殖における,遺伝情報の交換に相当する処理である。
交叉のアルゴリズムとしては一点交叉,二点交叉や一様交叉などが提案されている。

  \bunseki{※米村祥裕}
\subsubsection{一点交叉}
2つの個体を表すビット列をそれぞれ$s_1$,$s_2$とする。交叉を行う点をビット列中から
選択する。交叉を行う点以降のビット列について,$s_1$,と$s_2$とで交換する。これを一点交叉という。

  \bunseki{※米村祥裕}
\subsubsection{二点交叉}
一点交叉の場合は交叉を行う点を1箇所選んだが,二点交叉では2箇所選ぶ。選ばれた2点の間の
ビット列を交換する。

  \bunseki{※米村祥裕}
\subsubsection{一様交叉}
一様交叉ではまず遺伝子を表すビット列と同じ長さのビット列をランダムに生成する。
この時ランダムに生成されたビット列をマスクと呼ぶことにし,$i$番目のビットを$t_i \in {0, 1}$と表す。
$t_i = 1$であれば,二つの個体を表すそれぞれのビット列の$i$番目の要素を交換する。
$t_i = 0$であれば,交換を行わない。

一様交叉において$t_i$の内容によってどのように処理を行うかは実装によって異なることがある。

  \bunseki{※米村祥裕}
\subsubsection{突然変異}
交叉のみでは個体群の変化が小さく,局所的な解に陥ってしまうことがある。そのため,強引に異なる個体を生成する必要があり,
遺伝的アルゴリズムでは突然変異という処理によってこれを行う。具体的には,各個体の各ビット列上のビットに対して,一定の確率で
値を反転させる。これによってビット列の状態が確率的に変化する。

  \bunseki{※米村祥裕}