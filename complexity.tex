\section{複雑性を考慮した性能比較とその結果}

\subsection{各戦略の性能評価}

今回はシミュレーションから得た各戦略の勝率と定義した複雑性を用いて、各戦略の性能比較を行った。
性能の基準は以下の二通りを用意した。\\
~~ 性能1. (勝率) ÷ (複雑性)\\
~~ 性能2. (勝率) - (複雑性)\\
この評価基準に従って、デック数1の時の性能を表\ref{table:data_type4}、デック数無限の時の性能を表\ref{table:data_type5}にまとめた。

%\begin{table}[H]
%\caption{デック数1の時の各戦略の性能}
%\label{table:data_type4}
%\begin{center}
%\begin{tabular}{|c|c|c|c|c|c|}
%\hline
%戦略           & 圧縮長 & 勝率    & 複雑性   & 性能1  & 性能2   \\ \hline
%ベーシックストラテジー         & 30  & 0.431 & 0.375 & 1.14 & 0.052 \\ \hline
%ベーシックストラテジー改変1      & 28  & 0.429 & 0.35  & 1.22 & 0.076 \\ \hline
%ベーシックストラテジー改変2      & 26  & 0.430 & 0.325 & 1.21 & 0.072 \\ \hline
%15以上になるまでヒットする戦略 & 16  & 0.421 & 0.200 & 2.12 & 0.224 \\ \hline
%16以上になるまでヒットする戦略 & 16  & 0.416 & 0.200 & 2.07 & 0.214 \\ \hline
%17以上になるまでヒットする戦略 & 16  & 0.410 & 0.200 & 2.05 & 0.209 \\ \hline
%18以上になるまでヒットする戦略 & 16  & 0.421 & 0.200 & 1.97 & 0.193 \\ \hline
%\end{tabular}
%\end{center}
%\end{table}

%\begin{table}[H]
%\caption{デック数無限の時の各戦略の性能}
%\label{table:data_type5}
%\begin{center}
%\begin{tabular}{|c|c|c|c|c|c|}
%\hline
%戦略           & 圧縮長 & 勝率    & 複雑性   & 性能1  & 性能2   \\ \hline
%ベーシックストラテジー         & 30  & 0.427 & 0.375 & 1.14 & 0.052 \\ \hline
%ベーシックストラテジー改変1      & 28  & 0.424 & 0.35  & 2.12 & 0.224 \\ \hline
%ベーシックストラテジー改変2      & 26  & 0.414 & 0.325 & 1.07 & 0.214 \\ \hline
%15以上になるまでヒットする戦略 & 16  & 0.424 & 0.200 & 2.05 & 0.209 \\ \hline
%16以上になるまでヒットする戦略 & 16  & 0.414 & 0.200 & 1.97 & 0.193 \\ \hline
%17以上になるまでヒットする戦略 & 16  & 0.409 & 0.200 & 1.21 & 0.076 \\ \hline
%18以上になるまでヒットする戦略 & 16  & 0.393 & 0.200 & 1.21 & 0.072 \\ \hline
%\end{tabular}
%\end{center}
%\end{table}

\begin{table}[H]
\caption{デック数1の時の各戦略の性能}
\label{table:data_type4}
\begin{center}
\begin{tabular}{|c|c|c|c|c|c|}
\hline
戦略           & 圧縮長 & 勝率    & 複雑性   & 性能1  & 性能2   \\ \hline
ベーシックストラテジー         & 26  & 0.431 & 0.325 & 1.33 & 0.106 \\ \hline
ベーシックストラテジー改変1      & 24  & 0.429 & 0.300  & 1.43 & 0.129 \\ \hline
ベーシックストラテジー改変2      & 22  & 0.430 & 0.275 & 1.56 & 0.155 \\ \hline
15以上になるまでヒットする戦略 & 16  & 0.421 & 0.200 & 2.11 & 0.221 \\ \hline
16以上になるまでヒットする戦略 & 16  & 0.416 & 0.200 & 2.08 & 0.216 \\ \hline
17以上になるまでヒットする戦略 & 16  & 0.410 & 0.200 & 2.05 & 0.210 \\ \hline
18以上になるまでヒットする戦略 & 16  & 0.421 & 0.200 & 2.11 & 0.221 \\ \hline
\end{tabular}
\end{center}
\end{table}

\begin{table}[H]
\caption{デック数無限の時の各戦略の性能}
\label{table:data_type5}
\begin{center}
\begin{tabular}{|c|c|c|c|c|c|}
\hline
戦略           & 圧縮長 & 勝率    & 複雑性   & 性能1  & 性能2   \\ \hline
ベーシックストラテジー         & 26  & 0.427 & 0.325 & 1.31 & 0.102 \\ \hline
ベーシックストラテジー改変1      & 24  & 0.424 & 0.300  & 1.41 & 0.124 \\ \hline
ベーシックストラテジー改変2      & 22  & 0.414 & 0.275 & 1.51 & 0.139 \\ \hline
15以上になるまでヒットする戦略 & 16  & 0.424 & 0.200 & 2.12 & 0.224 \\ \hline
16以上になるまでヒットする戦略 & 16  & 0.414 & 0.200 & 2.07 & 0.214 \\ \hline
17以上になるまでヒットする戦略 & 16  & 0.409 & 0.200 & 2.05 & 0.209 \\ \hline
18以上になるまでヒットする戦略 & 16  & 0.393 & 0.200 & 1.97 & 0.193 \\ \hline
\end{tabular}
\end{center}
\end{table}

各戦略を比較し、次のような結果を得た。
勝率のみを考慮した場合、一定の数字以上になるまでヒットする戦略よりも、ベーシックストラテジーとそれを改変した戦略の方が有意に高い勝率だった。
ベーシックストラテジーと改変1、改変2のそれぞれの戦略間には有意な差が見られなかった。
複雑性を考慮して性能を評価した場合、デック数1の時は15以上になるまでヒットする戦略と、18以上になるまでヒットする戦略が最も性能が良かった。デック数無限の時は、15以上になるまでヒットする戦略が最も性能が良かった。


\bunseki{※渡邊凛}

\section{検証結果のまとめ}
これまで行った検証や性能評価による結果をまとめる。

勝率のみを考慮した場合
\begin{itemize}
\item 結果1:一定の数字以上になるあでヒットする戦略よりも、ベーシックストラテジーとベーシックストラテジー改変1、ベーシックストラテジー改変2のほうが有意に高い勝率だった
\item 結果2:ベーシックストラテジーとベーシックストラテジー改変1、ベーシックストラテジー改変2のそれぞれの戦略間に有意な差はみられなかった
\item 結果3:ベーシックストラテジー改変1と18以上までヒットする戦略にはデック数無限とデック数1で勝率に有意な差があった
\end{itemize}

勝率を考慮すると、ベーシックストラテジーとベーシックストラテジー改変1、ベーシックストラテジー改変2の3つが勝率が高く、優秀な戦略であることが分かった。ただ、ベーシックストラテジー改変1はデック数の違いによって勝率に差がある。

複雑性を考慮して性能を評価した場合
\begin{itemize}
\item 結果4:15以上になるまでヒットする戦略が1番優秀であることが判明した
\end{itemize}

複雑性を考慮すると、ベーシックストラテジーには改善の余地があることが判明した。
\bunseki{※柿崎大輝}