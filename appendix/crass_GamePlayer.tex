\begin{itemize}
\item ゲーム参加者を表すスーパークラス
\begin{lstlisting}
class GamePlayer:

    # 初期化関数
    def __init__(self):
        # 参加者の手札
        self.cards = []  
        # 参加者の手札の合計値
        self.total = 0  
        # 参加者の手札に含まれるAの枚数
        self.acetotal = 0  
        # 1として数えたAの枚数
        self.usedace = 0  
        # バーストしているかどうか
        self.burst = False  
        # ナチュラルブラックジャックを満たしているかどうか
        self.naturalbj = False 
        # 手札の合計値が21 になっているかどうか
        self.normalbj = False  

    # 子オブジェクトから呼び出せる初期化関数
    def initialize(self):
        self.cards = []
        self.total = 0
        self.acetotal = 0
        self.usedace = 0
        self.burst = False
        self.naturalbj = False
        self.normalbj = False

    # ゲームプレイヤーの手札の合計値を返す関数
    def totalvalue(self):
        i = 0
        self.total = 0
        self.acetotal = 0
        cardnum = len(self.cards)

        while i < cardnum:
            if (self.cards[i].rank == 'A'):
                self.acetotal += 1
            self.total += self.cards[i].value
            i += 1
        self.total -= 10 * self.usedace

        # プレイヤーのバースト判定の処理
        if (self.total > 21):
            if (self.acetotal - self.usedace > 0):
                self.total -= 10
                self.usedace += 1
                if (self.total > 21):
                    self.burst = True
            else:
                self.burst = True

\end{lstlisting}
\end{itemize}

\begin{itemize}
\item プレイヤークラス
\begin{lstlisting}
class Player(GamePlayer):
    # プレイヤーの初期化
    def __init__(self, name):
        self.name = name  # プレイヤーの名前
        self.totalwin = 0  # プレイヤーの勝利回数
        self.totallose = 0  # プレイヤーの敗北回数
        super().__init__()

    # プレイヤーがカードを受け取る時に使用する関数
    def dealedcard(self, card):
        self.cards.append(card)

    # プレイヤー側のヒットの処理
    def hit(self, dealer):
        self.dealedcard(dealer.dealcard())
        self.showhands()

    # プレイヤー側のスタンドの処理
    def stand(self):
        pass

    # プレイヤーの勝利回数を増やす
    def addtotalwin(self):
        self.totalwin += 1

    # プレイヤーの敗北回数を増やす
    def addtotallose(self):
        self.totallose += 1

\end{lstlisting}
\end{itemize}

\begin{itemize}
\item ディーラークラス(デック数有限)
\begin{lstlisting}
class Dealer(GamePlayer):
    # ディーラーの初期化
    def __init__(self, deckNum):
        self.deck = Deck(deckNum)
        # ディーラーがシャッフルする回数。今回は一万回シャッフルする。
        self.shufflenum = 10000
        self.deck.shuffle(deckNum * self.shufflenum)
        super().__init__()

    # カードを配る関数
    def dealcard(self):
        ''' デック数有限の際はこちらのコメントアウトを解除する '''
        card = self.deck.Cards[self.deck.current]
        self.deck.current += 1
        
        ''' デック数無限の際にはこちらのコメントアウトを解除する '''
        # randomcard = random.randrange(13);
        # card = Card(Card.RANKS[randomcard], Card.SUITS[0])
        # return card

    # 一番最初にカードを配る際の関数
    def firstdeal(self, player):
        super().__init__()
        for x in player:
            x.initialize()
        firstdeal = 2
        while firstdeal > 0:
            self.cards.append(self.dealcard())
            for x in player:
                x.cards.append(self.dealcard())
            firstdeal -= 1

    # 合計が17 を超えるまで引き続ける処理
    def continuehit(self):
        self.totalvalue()
        while (self.total < 17):
            self.cards.append(self.dealcard())
            self.totalvalue()

\end{lstlisting}
\end{itemize}

\begin{itemize}
\item ゲームマネージャークラス
\begin{lstlisting}
class GameManager:
    def __init__(self, players, dealer):
        self.players = players
        self.dealer = dealer
        self.checkdeal = True

    # 各プレイヤーとディーラーとの間で勝敗を決める
    def judge(self):
        for x in self.players:
            self.checkblackjack(x)
        self.checkblackjack(self.dealer)
        for player in self.players:
            if player.burst == True:
                player.addtotallose()
            elif player.burst == False and self.dealer.burst == True:
                player.addtotalwin()
            elif player.total > self.dealer.total:
                player.addtotalwin()
            elif player.total < self.dealer.total:
                player.addtotallose()
            elif player.total == self.dealer.total:
                if player.naturalbj and self.dealer.naturalbj:
                elif player.naturalbj and self.dealer.normalbj:
                    player.addtotalwin()
                elif player.normalbj and self.dealer.naturalbj:
                    player.addtotallose()
                elif player.normalbj and self.dealer.normalbj:
                    pass
                else:
                    pass
\end{lstlisting}
\end{itemize}